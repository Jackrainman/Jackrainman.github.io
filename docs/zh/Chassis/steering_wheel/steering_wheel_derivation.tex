\documentclass[12pt,a4paper]{article}
\usepackage{geometry}
\usepackage{amsmath}
\usepackage{amssymb}
\usepackage{graphicx}
\usepackage{float}
\usepackage{booktabs}
\usepackage{ctex}

\geometry{margin=2.5cm}
\title{\textbf{舵轮底盘运动学解算推导}}
\author{基于 steering_wheel.c/h 代码分析}
\date{\today}

\begin{document}
\maketitle

\section{引言}
舵轮底盘由多个独立的舵向轮模块组成,每个模块包含:
\begin{itemize}
    \item \textbf{舵向电机}:控制轮子转向角度
    \item \textbf{航向电机}:控制轮子转动速度
\end{itemize}

舵轮底盘可实现全方位移动(omnidirectional movement),包括平动、自转以及两者的组合。

\section{系统建模}

\subsection{坐标系定义}

\begin{enumerate}
    \item \textbf{世界坐标系} $\{O_w\}$:全局固定坐标系
    \item \textbf{车身坐标系} $\{O_b\}$:固定在底盘中心的坐标系
    \item \textbf{单个舵轮}:具有独立的转向和驱动能力
\end{enumerate}

\subsection{四轮底盘布局}
按照代码中的定义,四轮底盘呈矩形布局,轮子到中心的距离(半径)为 $r$。

\subsection{三轮底盘布局}
三轮底盘呈等边三角形布局,各轮间隔 $120°$。

\section{速度分解与合成}

\subsection{平动分量}
设车身坐标系下的目标速度为 $(V_x, V_y)$,车身相对于世界坐标系的偏航角为 $\psi$(逆时针为正)。

世界坐标系下的速度通过坐标变换得到:
\begin{equation}
\begin{bmatrix} V_{wx} \ V_{wy} \end{bmatrix} = 
\begin{bmatrix}
\cos\psi & -\sin\psi \
\sin\psi & \cos\psi
\end{bmatrix}
\begin{bmatrix} V_x \ V_y \end{bmatrix}
\end{equation}

代码中使用负的偏航角实现变换(line 223-226):
\begin{align}
mix\_x[i] &= V_x \cos\psi + V_y \sin\psi + speed\_wx[i] \
mix\_y[i] &= V_y \cos\psi - V_x \sin\psi + speed\_wy[i]
\end{align}

\subsection{转动分量}
当底盘以角速度 $\omega$ 自转时,各轮产生的线速度为:
\begin{equation}
\vec{v}_{\omega,i} = \omega r \cdot (-\sin\alpha_i, \cos\alpha_i)
\end{equation}

其中 $\alpha_i$ 是第 $i$ 个轮子与中心连线相对于 $x$ 轴的夹角。

\subsubsection{四轮底盘(line 196-204)}
各轮位于 $\pm 45°$ 方向:
\begin{align}
speed\_wx &= \omega \cdot \left[\frac{\sqrt{2}}{2}, \frac{\sqrt{2}}{2}, -\frac{\sqrt{2}}{2}, -\frac{\sqrt{2}}{2}\right] \
speed\_wy &= \omega \cdot \left[-\frac{\sqrt{2}}{2}, \frac{\sqrt{2}}{2}, -\frac{\sqrt{2}}{2}, \frac{\sqrt{2}}{2}\right]
\end{align}

\subsubsection{三轮底盘(line 250-256)}
各轮位于 $0°, 120°, -120°$ 方向:
\begin{align}
speed\_wx &= \omega \cdot \left[1, -\frac{1}{2}, -\frac{1}{2}\right] \
speed\_wy &= \omega \cdot \left[0, -\frac{\sqrt{3}}{2}, \frac{\sqrt{3}}{2}\right]
\end{align}

\subsection{速度矢量叠加}
每个轮子的合速度(line 224-226):
\begin{align}
mix\_x[i] &= V_{wx} + speed\_wx[i] \
mix\_y[i] &= V_{wy} + speed\_wy[i]
\end{align}

\section{轮子目标状态计算}

\subsection{速度大小(line 228)}
\begin{equation}
V_i = \sqrt{(mix\_x[i])^2 + (mix\_y[i])^2}
\end{equation}

\subsection{角度计算(line 238-240)}
极坐标映射将角度映射到 $[-\pi, \pi]$ 范围:
\begin{equation}
\theta_i = \begin{cases}
\arccos\left(\dfrac{mix\_y[i]}{V_i}\right) & \text{if } mix\_x[i] < 0 \
-\arccos\left(\dfrac{mix\_y[i]}{V_i}\right) & \text{if } mix\_x[i] \geq 0
\end{cases}
\end{equation}

\section{最小转角优化算法(line 121-160)}

舵向电机有两个可选方向:$\theta$ 或 $\theta + \pi$(后者需反转航向电机)。

\subsection{角度差计算}
设当前角度为 $\theta_{curr}$:
\begin{align}
\Delta\theta_1 &= \text{remap}(\theta - \theta_{curr}) \
\Delta\theta_2 &= \text{remap}(\theta + \pi - \theta_{curr})
\end{align}

其中重映射函数(line 106-112)将角度映射到 $[-\pi, \pi]$:
\begin{equation}
\text{remap}(\alpha) = \begin{cases}
\alpha - 2\pi & \text{if } \alpha > \pi \
\alpha + 2\pi & \text{if } \alpha < -\pi \
\alpha & \text{otherwise}
\end{cases}
\end{equation}

\subsection{优化选择(line 145-157)}
\begin{equation}
(\theta_{final}, V_{final}) = \begin{cases}
(\theta + \pi, -V) & \text{if } |\Delta\theta_2| < |\Delta\theta_1| \
(\theta, V) & \text{otherwise}
\end{cases}
\end{equation}

\section{算法流程总结}

\begin{enumerate}
    \item \textbf{输入}:$(V_x, V_y, \omega)$ 和车身偏航角 $\psi$
    \item \textbf{坐标变换}:将速度转换到世界坐标系
    \item \textbf{分解转动}:计算自转在各轮产生的线速度
    \item \textbf{矢量叠加}:平动与转动速度合成
    \item \textbf{计算模值}:求各轮目标速度大小
    \item \textbf{计算角度}:求各轮目标转向角度
    \item \textbf{最小转角}:选择最优转向路径
    \item \textbf{输出}:各轮目标角度和速度
\end{enumerate}

\section{关键公式汇总}

\begin{align}
&\text{坐标变换:} && V_{wx} = V_x\cos\psi - V_y\sin\psi \
&                && V_{wy} = V_x\sin\psi + V_y\cos\psi \[8pt]
&\text{转动线速度:} && v_\omega = \omega r \[8pt]
&\text{速度合成:} && \vec{V} = \vec{V}_{trans} + \vec{V}_{rot} \[8pt]
&\text{速度大小:} && V = \sqrt{V_x^2 + V_y^2} \[8pt]
&\text{角度计算:} && \theta = \text{atan2}(V_y, V_x) \[8pt]
&\text{角度映射:} && \text{remap}(\theta) = \text{fmod}(\theta, 2\pi) \pm 2\pi
\end{align}

\end{document}
